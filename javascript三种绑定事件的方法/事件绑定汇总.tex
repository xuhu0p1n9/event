那么既然已经有两种绑定事件的方法为什么还要有第三种呢?答案是这样的:

用 "addeventlistener" 可以绑定多次同一个事件,且都会执行,而在DOM结构如果绑定两个 "onclick" 事件,只会执行第一个;在脚本通过匿名函数的方式绑定的只会执行最后一个事件。

addEventListener()和attachEvent()的函数语法:
1 elementObject.addEventListener(eventName,handle,useCapture) (支持主流浏览器、以及IE9.0及以上)
eventName:要绑定的事件名称。注意写法,比如点击事件,写成click,而不是onclick.
handle: 处理事件的函数名。但是写法上没有小括号。
useCapture:Boolean类型,是否使用捕获,一般用false,具体涉及到的会在后边总结。
2 elementObject.attachEvent(eventName,handle);(仅支持IE8及以下)
从网上找到了一个兼容的好办法,相比较if。。else语句,这个方法用的是try..catch错误处理语句,可以避免浏览器出现错误提示。

function addEvent(obj,type,handle){
  try{
   obj.addEventListener(type,handle,false);
  }catch(e){
   try{
    obj.attachEvent('on'+type,handle);
   }
   catch(e){
    obj['on' + type]=handle;//早期浏览器
   }
  }
}